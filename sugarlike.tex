\documentclass[11pt]{article}

\title{\textbf{Sugarlike: A Language Identification Library for Low Resource Languages}}
\author{Susanne Fertmann, Guy Emerson, Liling Tan}
\date{}

\begin{document}

\maketitle

\section{Introduction}

Sugarlike is a language identification library designed for identifying low resource language given a document\footnote{A document can refer to a webpage, a textfile or even a sentence.}.

@Su-Ga: I can't think of a name, so i anyhow gave it one "sugarlike". Please feel free to change it

\subsection{Previous Work}

Previous work on language identification for low resource languages...
Kevin Schnell's work, blah blah blah... full data not available but the 3-grams are good source of information.

\section{Data}
With regards to the data sources, we propose crawling Twitter posts using a fixed list of hashtags in English and their respective translations. Although the redistribution of our crawled corpus remain questionable, we will distribute the source code that allows researchers to rebuild an update corpus. [we need some proper methodology... ahahhaa]

\section{Model}

I am not very sure, lol - LL

\subsection{Need for Modularity}

The trained language identification model would remain open and free for use. And more importantly, the model we are building must remain modular in the sense that other research can rebuild an updated Twitter corpus and extend the model simply by either adding new features or new datapoints to the classification model.


\section{Implementation}

Build a crawl for Twitter post.

\end{document}
