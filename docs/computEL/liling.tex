% \documentclass[11pt]{article}
% \title{\textbf{universal corpus}}
% \date{}
% \begin{document}
% \maketitle

\section{Corpus Coverage and Universality, something, something numbers...}

\begin{table}[h!]
\centering
    \begin{tabular}{l|rrrr}
    ~         				& \#Languages & \#Families 	&\#tokens		& Size	\\ \hline
    ODIN      				& 1,217      & 101       		& 58,556		& 39 MB		\\
    Omniglot  				& 134        & 21        		&	6,749			& 677 KB	\\
    UDHR      				& 355        & 47        		&	33,117		& 5.2 MB	\\
    Wikipedia 				& 209        & 22       		&						& 37 GB		\\ \hline
    \textbf{Combined}	& 1,347			 & 106 
    \end{tabular}
\caption{Corpus Coverage}
\end{table}

\noindent
According to ethnologue \\
Total Living languages: 7105 \\
Total Living language families: 147 \\

\begin{table}[h!]
    \begin{tabular}{l|ccc|ccc}
    ~        & complete & ~       & ~       & ward    & ~       & ~       \\ \cline{2-7}
    ~        & precision & recall       & f-score       & precision    & recall       & f-score      \\ \hline
    distance & 0.0614   & 0.8565 & \textbf{0.1099} & 0.06140  & 0.8565 & \textbf{0.1099} \\
    maxclust & 0.1925  & \textbf{0.0927} & 0.0692 & \textbf{0.1963} & 0.0905 & 0.0686  \\
    \end{tabular}
\caption{Comparing of clustering algorithms where number of clusters is set to genetic grouping on Ethnologue.}
\end{table}

\noindent
see full table in http://goo.gl/bs2vPo


%\end{document}
