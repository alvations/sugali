\documentclass[11pt]{article}
\usepackage{acl2014}
\usepackage{times}
\usepackage{url}
\usepackage{latexsym}

\bibliographystyle{acl}

\usepackage[utf8]{inputenc}
\usepackage{graphicx}

\usepackage{enumitem}
\usepackage{listings}
\usepackage{fancyvrb}
\usepackage{color}
\definecolor{gray}{rgb}{0.4,0.4,0.4}
\definecolor{darkblue}{rgb}{0.0,0.0,0.6}
\definecolor{cyan}{rgb}{0.0,0.6,0.6}
\definecolor{orange}{rgb}{1,0.5,0}

\lstset{
  basicstyle=\ttfamily,
  columns=fullflexible,
  showstringspaces=false,
  commentstyle=\color{gray}\upshape
}

\lstdefinelanguage{XML}
{
  morestring=[b]",
  morestring=[s]{>}{<},
  morecomment=[s]{<?}{?>},
  stringstyle=\color{black},
  identifierstyle=\color{darkblue},
  keywordstyle=\color{cyan},
  morekeywords={xmlns,version,type}% list your attributes here
}


\setlength\titlebox{5cm}

\title{SeedLing: Building and using a seed corpus \\ for the Human Language Project}
%Getting the Ball Rolling: Producing a Foundation Text for a Universal Corpus of the World's Languages}

% \author{Liling Tan, Guy Emerson, Susanne Fertmann, Alexis Palmer and Michaela Regneri \\
%   Universität des Saarlandes \\
%   Campus, 66123 Saarbrücken, Germany \\
%   {\tt liling.tan@uni-saarland.de, emerson@coli.uni-saarland.de,} \\
%   {\tt susfert@coli.uni-saarland.de}}

\date{}

\begin{document}
\maketitle
\begin{abstract}
\textcolor{blue}{[revisit abstract]}
Computational study of endangered languages is constrained by the lack of digitally-available data. Existing corpora are limited in the range of languages covered, in standardisation, or in machine-readability.
This makes the situation even worse for the computational linguist, especially one who would like to take a cross-linguistic or typological approach. We first survey existing efforts to compile cross-linguistic resources, then describe our own approach and give an example application - language clustering. To build the foundation text for a Universal Corpus, we crawled and cleaned texts from several web sources that contain data from a large number languages, and converted them into a standardised form consistent with the guidelines set out by \newcite{abney2010universal}. The resulting corpus is more easily-accessible and machine-readable than any of the underlying data sources, and represents a significant base corpus for researchers to draw on and add to in the future.
\end{abstract}


\section{Introduction} \label{sec:intro}

At the time of writing, 7105 living languages are documented in Ethnologue\footnote{http://www.ethnologue.com}, but \newcite{krauss1992crisis} estimated 50\% of these are not being learnt by new generations of speakers, and risk extinction by the end of the century. However, only a fraction of the world's languages are well documented, fewer have machine-readable resources, and fewer again have resources with linguistic annotations \cite{maxwell2006annotation} - so the time to work on compiling these resources is now.

Although there have been some previous attempts to produce cross-linguistic resources, there are few which both cover a wide range of languages, and are also machine-readable. We survey existing efforts in section \ref{sec:related}, and discuss their limitations in more detail.

Abney and Bird \shortcite{abney2010universal,abney2011data} posed the grand challenge of building a Universal Corpus, calling it the Human Language Project. Such a corpus would include all of the world's languages, in a consistent structure, facilitating large-scale cross-linguistic processing. They propose a specific data structure, which we describe and discuss in section \ref{sec:structure}. We have accepted their challenge, and have begun converting existing resources into a format consistent with their specifications for a universal corpus. We have drawn on four [or five?] web sources, cleaning and standardising them as described in section \ref{sec:sources}, to produce a seed corpus for the Human Language Project. In sections \ref{sec:stats} and \ref{sec:copyright}, we respectively give a summary of the data contained, and discuss copyright and distribution. \textcolor{blue}{[needs rewriting following reorganization.]}

[Note: remove this section if we don't do the work!] We believe the resulting corpus is the first of its kind: large enough and consistent enough to allow language processing on a grand scale. In section \ref{sec:cluster}, we give an example application of this corpus: language clustering. We use the frequencies of character n-grams and words to estimate the similarity of two languages. Despite this approach being highly dependent on orthography, we are able to reconstruct substantial parts of several language families, demonstrating the utility of this resource in cross-linguistic research. Finally, we discuss future work in section \ref{sec:future}, and conclude in section \ref{sec:conclusion}.


\section{Related Work} \label{sec:related}

In this section, we review existing efforts to compile multilingual machine-readable resources.  Although some commercial resources are available, we restrict attention to freely accessible data\footnote{All figures given below were correct at the time of writing, but it must be borne in mind that most resources discussed below are constantly growing.}.

\paragraph{Traditional archives.}
Many archives exist to store the wealth of traditional resources produced by the documentary linguistics community.  Such documents are increasingly being digitised, or produced in a digital form, and there are a number of archives which now offer free online access to their data.

Some archives aim for a universal scope, including languages from all parts of the world.  Of particular note are: The Language Archive, maintained by the Max Planck Institute of Psycholinguistics, which includes DoBeS (Dokumentation Bedrohter Sprachen) and many other projects;
Collection Pangloss, maintained by LACITO (Langues et Civilisations à Tradition Orale);
The Endangered Languages Archive (ELAR), maintained by the School of Oriental and African Studies, which forms one part of the Hans Rausing Endangered Languages Project (HRELP).

Regional archives include: AILLA (Archive of Indigenous Languages of Latin America), University of Texas at Austin;
AIATSIS collections (Australian Institute of Aboriginal and Torres Strait Islander Studies);
California Language Archive, managed by the Survey of California and Other Indian Languages;
ANLA (Alaska Native Language Archive), at the University of Alaska Fairbanks;
PARADISEC (Pacific and Regional Archive for Digital Sources in Endangered Cultures).

However, there are two main problems common to all of the above data sources.  Firstly, the data is not always machine readable.  Even where the data is available digitally, these often take the form of scanned images or audio files.  While both can provide invaluable information, they are extremely difficult to process with a computer, requiring an impractical level of image or video pre-processing before linguistic analysis can begin.  Even textual data, which avoids these issues, may not be available in a machine-readable form, being stored as pdfs or other opaque formats.
Secondly, when data is machine readable, the format can vary wildly.  This makes automated processing difficult, especially if one is not aware of the details of each project.  Even when metadata standards and encodings agree, there can be idiosyncractic markup or non-linguistic information, such as labels for speakers in the transcript of a conversation.

We can see that there is still much work to be done by individual researchers in digitising and standardising linguistic data, and it is outside of the scope of this paper to attempt this for the above archives.  Guidelines for producing new materials are available from the E-MELD project (Electronic Metastructure for Endangered Languages Data), which specifically aimed to deal with the expanding number of standards for linguistic data.  It gives best practice recommendations, illustrated with eleven case studies, and provides input tools which link to the GOLD ontology language, and the OLAC metadata set.  Further recommendations are given by \newcite{bird2003port}, who describe seven dimensions along which the portability of linguistic data can vary. Various tools are available from The Language Archive at the Max Planck Institute for Psycholinguistics.

Many of these archives are part of the Open Language Archive Community (OLAC), a subcommunity of the Open Archives Initiative.  OLAC maintains a metadata standard, based on the 15-element Dublin Core, which allows a user to search through all participating archives in a unified fashion.  However, centralising access to disparate resources, while of course extremely helpful, does not solve the problem of inconsistent standards.  Indeed, it can be difficult even to answer simple questions like ``how many languages are represented?''

Such resources are invaluable for many purposes. For instance, there can be plenty of material to work with for a
researcher who is looking to find data on a specific language, who is willing to put in a little time browsing through different catalogues, and who only needs human-readable data.  However, for large-scale machine processing, they leave much to be desired.


\paragraph{Generic corpus collections.}
Some corpus collections exist which do not focus on endangered languages, but which nonetheless cover an increasing number of languages.

MetaShare (Multilingual Europe Technology Alliance) provides data in a little over 100 languages. While language codes are used, they have not been standardised, so that multiple codes are used for the same language.  Linguistic Data Consortium (LDC) and the European Language Resources Association (ELRA) both offer data in multiple languages.  However, while large in size, they cover only a limited number of languages.  Furthermore, the corpora they contain are stored separately, making it difficult to access data according to language.

% It seems likely that resources which do not specifically aim to cover endangered languages will continue to be skewed heavily towards a handful of high-resource languages.

\paragraph{Parallel corpora.}

The Machine Translation community has assembled a number of parallel corpora, which are crucial for statistical machine translation. The OPUS corpus \cite{tiedemann2012opus} subsumes a number of other well-known parallel corpora, such as Europarl, and covers documents from 350 languages, with various language pairs.  

%However, not all of the data has been manually checked, which introduces noise.

\paragraph{Web corpora.}

There has been increasing interest in deriving corpora from the web, due to the promise of large amounts of data.  The majority of web corpora are however aimed at either one or a small number of languages, which is perhaps to be expected, given that the majority of online text is written in a handful of high-resource languages.  Nonetheless, there have been a few efforts to apply the same methods to a wider range of languages.

HC Corpora currently provides download of corpora in 68 different language varieties, which vary in size from 2M to 150M words. The corpora are thus of a respectable size, but only 1\% of the world's languages are represented.  A further difficulty is that languages are named, without the corresponding ISO language codes.

The Leipzig Corpora Collection (LCC)\footnote{\url{http://corpora.uni-leipzig.de}} \cite{biemann2007leipzig} provides download of corpora in 117 languages, and dictionaries in a number of others, bringing the total number of represented languages up to 230. The corpora are large, readily available, in plain text, and labelled with ISO language codes.

The Crúbadán Project aims to crawl the web for text in low-resource languages, and data is currently available for 1872 languages.  This represents a significant portion of the world's languages; unfortunately, due to copyright restrictions, only lists of n-grams and their frequencies are publically available, not the texts themselves.  While the breadth of languages covered makes this a useful resource for cross-linguistic research, the lack of actual texts means that only a limited range of applications are possible with this data.

% While the above efforts are commendable, and allow cross-linguistic analysis to a certain extent, there is still a long way to go to producing a universal corpus.

\paragraph{Cross-linguistic projects.}

Responding to the call to document and preserve the world's languages, highly cross-linguistic projects have sprung up, striving towards the aim of universality.  Of particular note are the Endangered Languages Project, and the Rosetta Project. These projects are to be praised for their commitment to universality, but in their current forms it is difficult to use their data to perform large-scale NLP.  

% A researcher looking for a data on a specific language can easily navigate to the language in question, and download the available resources - but trying to access all the data can be awkward.  While the sleek online interface used by the Endangered Languages Project may make it more accessible to those unfamiliar with language processing, it is not ideal for a computational linguistics researcher.

%\documentclass[11pt]{article}
\title{\textbf{universal corpus}}
\date{}
\begin{document}
\maketitle

\section{Corpus Coverage and Universality, something, something numbers...}

\begin{table}[h!]
\centering
    \begin{tabular}{l|rrrr}
    ~         				& \#Languages & \#Families 	&\#tokens		& Size	\\ \hline
    ODIN      				& 1,217      & 101       		& 58,556		& 39 MB		\\
    Omniglot  				& 134        & 21        		&	6,749			& 677 KB	\\
    UDHR      				& 355        & 47        		&	33,117		& 5.2 MB	\\
    Wikipedia 				& 209        & 22       		&						& 37 GB		\\ \hline
    \textbf{Combined}	& 1,347			 & 106 
    \end{tabular}
\caption{Corpus Coverage}
\end{table}

\noindent
According to ethnologue \\
Total Living languages: 7105 \\
Total Living language families: 147 \\

\begin{table}[h!]
    \begin{tabular}{l|ccc|ccc}
    ~        & complete & ~       & ~       & ward    & ~       & ~       \\ \cline{2-7}
    ~        & precision & recall       & f-score       & precision    & recall       & f-score      \\ \hline
    distance & 0.0614   & 0.8565 & \textbf{0.1099} & 0.06140  & 0.8565 & \textbf{0.1099} \\
    maxclust & 0.1925  & \textbf{0.0927} & 0.0692 & \textbf{0.1963} & 0.0905 & 0.0686  \\
    \end{tabular}
\caption{Comparing of clustering algorithms where number of clusters is set to genetic grouping on Ethnologue.}
\end{table}

\noindent
see full table in http://goo.gl/bs2vPo


\end{document}

\section{The Data}\label{sec:data}

\subsection{Representation and Universality} \label{sec:stats}

According to Ethnologue, there are 7105 living languages, and 147 living language families. Across all our data sources, we manage to cover 1347 languages in 106 families, which represents 19.0\% of the world's languages. To get a better idea of the kinds of languages represented, we give a breakdown according to their EGIDS scores (Expanded Graded Intergenerational Disruption Scale) in figure \textcolor{blue}{[insert reference here when it's ready]}. The values in each cell have been colored according to proportion of languages represented, with green indicated good coverage and red poor. It's interesting to note that vigorous languages (6a) are poorly represented across all data sources, and worse than more endangered categories. In terms of language documentation, vigorous languages are less urgent goals than those in categories 6b and up, but this highlights an unexpected gap in linguistic resources.

%\begin{table}
%\centering
%    \begin{tabular}{l|cc}
%    ~         				& \#Languages & \#Families \\ \hline
%    ODIN      				& 1218       & 101       \\
%    Omniglot  				& 134        & 21        \\
%    UDHR      				& 355        & 47        \\
%    Wikipedia 				& 209        & 22       \\ \hline
%    \textbf{Combined}	& 1347			 & 106 
%    \end{tabular}
%\caption{Corpus Coverage}
%\end{table}

\begin{table}[h!]
\centering
    \begin{tabular}{l|rr|rr}
    ~         				& \#Languages & \#Families 	&\#tokens		& Size	\\ \hline
    ODIN      				& 1,217      & 101       		& 58,556		& 39 MB		\\
    Omniglot  				& 134        & 21        		&	6,749			& 677 KB	\\
    UDHR      				& 355        & 47        		&	33,117		& 5.2 MB	\\
    Wikipedia 				& 209        & 22       		&						& 37 GB		\\ \hline
    \textbf{Combined}	& 1,347			 & 106 
    \end{tabular}
\caption{Corpus Coverage}
\label{table:corpus}
\end{table}

\textcolor{blue}{[insert heatmap here]}


\subsection{Universal Corpus and Data Structure} \label{sec:structure}

Building on their previous paper, \newcite{abney2011data} describe the data structure they envisage for the universal corpus in more detail, distinguishing between \textbf{aligned texts} and \textbf{analysed texts}. Aligned texts consist of multiple parallel documents, aligned at the document, sentence, or word level. The collection of parallel documents as a whole is assigned a unique identifier, with each individual document labelled by this identifier and the corresponding language code.  A monolingual document is simply one for which the document ID only has a single language code associated with it.  Sentences and words within documents are also assigned identifiers (relative to the document), if alignment has been performed at that level.

Analyses texts, in addition to the raw text, contain more detailed annotations including parts of speech, morphological information, and syntactic relations.  This is stored as a table, where each row represents a word, and each column a type of information.  They propose the following columns: document ID, language code, sentence ID, word ID, wordform, lemma, morphological information, part of speech, gloss, head/governor, and relation/role.

Out of our data sources, three can be straightforwardly represented in their aligned text structure.  However, ODIN contains richer annotations, which are in fact difficult to fit into their proposal, which we discuss in section \ref{sec:odin} below.


\subsection{Data Sources} \label{sec:sources}

Although data size matters in general NLP, \emph{universality} is the utmost priority for a universal corpus. We chose to focus on the following data sources, because they include a large number of languages, include several parallel texts, and demonstrate a variety of data types which a linguist might encounter (structured, semi-structured, unstructured):

\begin{enumerate}
\item The Online Database of Interlinear Text (ODIN)
\item The Omniglot website
\item The Universal Declaration of Human Rights (UDHR)
\item Wikipedia
\end{enumerate}


Our resulting corpus runs the full gamut of text types outlined by Abney and Bird, ranging from single-language text (Wikipedia) to parallel text (UDHR and Omniglot) to IGTs (ODIN).  We describe each source in the following subsections.


\subsubsection{Universal Declaration of Human Rights}

The Universal Declaration of Human Rights (UDHR) is a document released by the United Nations in 1948, and represents the first global expression of human rights. It consists of 30 articles, amounting to about four pages of text. This is a useful document for CL, since it has been translated into a wide variety of languages, providing a highly parallel text.


\subsubsection{Wikipedia}

Wikipedia\footnote{url{http://www.wikipedia.org}} is a collaboratively-edited encyclopedia, appealing to use for NLP because of its large size and easy availability. At the time of writing, it contained 30.8 million articles in 287 languages, which provides a sizeable amount of clean monolingual text in a fairly wide range of languages. Text dumps are made regularly, and can be downloaded from \url{www.dumps.wikimedia.org}.

\subsubsection{Omniglot}

The Omniglot website\footnote{url{http://www.omniglot.com}} is an online encyclopedia of writing systems and languages. We chose to extract information from pages on \emph{`Useful foreign phrases'} and the \emph{`Tower of Babel'} story, both of which give us a parallel data in a reasonably large number of languages. The urls for these resources are:

\begin{itemize}[noitemsep]
\item \url{www.omniglot.com/language/phrases/*} 
\item \url{www.omniglot.com/babel/*}
\end{itemize}

\subsubsection{ODIN} \label{sec:odin}

ODIN (The Online Database of Interlinear Text) is a repository of interlinear glossed texts (IGTs) extracted from scholarly documents \cite{lewis2006odin,lewis2010odin}.  Compared to other resources, it is notable for the breadth of languages included, and the level of linguistic annotation.  An IGT consists of three lines: (i) the source, a sentence in a target language, (ii) the gloss, an analysis of each element in the source, and (iii) the translation, done at the sentence level. The gloss line can additionally include a number of linguistic terms, which means that the gloss can more properly described as being written in a metalanguage, rather than a natural language.  In ODIN, translations are into English, and glosses are written in an English-based metalanguage.  An accepted set of guidelines are given by the Leipzig Glossing Rules,\footnote{\url{http://www.eva.mpg.de/lingua/resources/glossing-rules.php}}, where morphemes within words are separated by hyphens (or equal signs, for clitics), and the same number of hyphens should appear in each word of the source and gloss.

In contrast, the proposed data structure for the universal corpus is aligned at the word level, and includes a specific list of relevant features which should be used to annotate words. When we try to adapt IGTs into this format, we run into certain problems.  Firstly, there is the problem that the most fundamental unit of analysis according to the Leipzig Glossing Rules is the morpheme, not the word.  Ideally, we should encode this information explicitly in a universal corpus, assigning a unique identifier to each morpheme (instead of, or in addition to each word). Indeed, \newcite{haspelmath2011segment} argues that there is no cross-linguistically valid definition of "word", which undermines the central position of words in the proposed data structure.

Secondly, it is unclear how to represent the gloss.  Since the gloss line is written in a natural language, we cannot treat it as a simple translation.  However, it is not straightforward to incorporate it into the proposed structure for analysed texts, either.  We could simply always map the gloss to the \textsc{gloss} field, but this would run against the suggestion that this field should "contain symbols from controlled vocabularies".  Indeed, this information might be more suited to the \textsc{morph, pos,} or \textsc{rel} fields, but it would be difficult to automatically process ODIN to determine which morphemes should be assigned to which field.

One possible resolution is to move all elements of the gloss written in capital letters to the \textsc{morph} field (as functional elements are usually annotated in this way), and all remaining elements to the \textsc{gloss} field.  However, this loses information, since we no longer no which morpheme has which meaning.  It would be ridiculous to throw away information in the IGTs, but to keep all information, we need to modify the \newcite{abney2011data}'s proposal.

The simplest solution we can see is to allow morphemes to be a level of structure in the universal corpus, just as documents, sentences, and words already are.  The overall architecture remains unchanged.  We must then decide how to represent the glosses.

Even though glosses in ODIN are based on English, having been extracted from English-language documents, this is not true of IGTs in general.  In particular, it is common for documentary linguists working on indigenous languages of the Americas to provide glosses and translations based on Spanish.  For this reason, we believe it would be wise to specific the language used to produce the gloss.  Since it is not quite the language itself, but a metalanguage, one solution would be to use new language codes that make it clear both that a metalanguage is being used, and also what natural language it is based on.  The five-letter code \texttt{gloss} cannot be confused with any code in any version of ISO 639 (which includes codes of length two to four).  Following the convention that subvarieties of a language are indicated with suffixes, we can append the code of the natural language being used.  For example, glosses into English and Spanish-based metalanguages would be given the codes \texttt{gloss-eng} and \texttt{gloss-spa}, respectively.

One benefit of this approach is that glossed texts are treated in exactly the same way as parallel texts.  There is a unique identifier for each morpheme, and glosses are stored under this identifier and the corresponding gloss code.  Furthermore, to motivate the important place of parallel texts in a universal corpus, Abney and Bird view translations into a high-resource reference language as a convenient surrogates of meaning. By the same reasoning, we can use glosses to provide a more detailed surrogate of meaning, only written in a metalanguage instead of a natural one.

\section{Data Clean-Up, Consistency, and Standardisation} \label{sec:case_studies}
Consistency in data structures and formatting is essential to facilitate use of data in computational linguistics research \cite{palmer2010lilt}. In the following subsections, we describe the processing required to convert the data into a standardised form.  We then discuss standardisation of language codes and file formats.

\subsection{Case Studies}


\paragraph{UDHR.} We used the plain-text UDHR files available from the Unicode website\footnote{\url{http://unicode.org/udhr/d}} which uses UTF-8 encoding for all languages. The first four lines of each file record metadata, and the rest is the translation of the UDHR. This dataset is extremely clean, and simply required segmentation into sentences.


\paragraph{Wikipedia.}
One major issue with using the Wikipedia dump is the problem of separating text from abundant source-specific markup. To convert compressed Wikipedia dumps to textfiles, we used the WikiExtractor\footnote{\url{http://medialab.di.unipi.it/wiki/Wikipedia_Extractor}} tool. After conversion into textfiles, we used several regular expressions to delete residual Wikipedia markup and so-called ``\emph{magic words}".\footnote{\url{http://en.wikipedia.org/wiki/Help:Magic_words}}


\paragraph{Omniglot.}
The main issue with extracting the Omniglot data is that the pages are designed to be human-readable, not machine-readable.  Cleaning this data required parsing the HTML source, and extracting the relevant content, which required different code for the two types of page we considered ('\emph{Useful foreign phrases}' and '\emph{Tower of Babel}').  Even after automatic extraction, some noise in the data remained, such as explanatory notes given in parentheses, which are written in English and not the target language.  Even though the total amount of data here is small compared to our other sources, the amount of effort required to process it was not, because of these idiosyncracies.  We expect that researchers seeking to convert data from human-readable to machine-readable formats will encounter similar problems, but unfortunately there is unlikely to be a one-size-fits-all solution to this problem.


\paragraph{ODIN.}
The ODIN data is easily accessible in XML format from the online database\footnote{\url{http://odin.linguistlist.org/download}}. Data for each language is saved in a separate XML file and the IGTs are encoded in tags of the form \texttt{<igt><example>...</example></igt>}.  For example, the IGT in Figure~\ref{fig:odin_fijian} is represented by the XML snippet in Figure~\ref{fig:odin_fijian_xml}.

The primary problem in extracting the data is a lack of consistency in the IGTs. In the above examples, the sentence is introduced by a letter or number, which needs to be removed; however, the form of such indexing elements varies. In addition, the source line in Figure~\ref{fig:odin_yimas_xml} includes two types of metadata: the language name, and a citation, both of which introduce noise.  Finally, extraneous punctuation such as the quotation marks in the translation line need to be removed. We used regular expressions for cleaning lines within the IGTs.

\begin{figure}[t]
\quad 21 a.\quad o lesu mai \\
\indent \qquad\qquad 2sg return here \\
\indent \qquad\qquad `\emph{You return here.}' \\
\caption{Fijian IGT from ODIN} \label{fig:odin_fijian}
\end{figure}


\begin{figure}[t]
\small
\begin{lstlisting}[language=XML]
<igt>
  <example>
    <line>21 a. o lesu mai</line>
    <line>2sg return here</line>
    <line>`You return here.'</line>
  </example>
</igt>
\end{lstlisting} 
\caption{Fijian IGT in ODIN's XML format} \label{fig:odin_fijian_xml}
\end{figure}


\begin{figure}[t]
\small
\begin{lstlisting}[language=XML]
<igt>
  <example>
    <line>(69) na-Na-tmi-kwalca-t 
    Yimas (Foley 1991)</line>
    <line>3sgA-1sgO-say-rise-PERF
    </line>
    <line>`She woke me up' 
    (by verbal action)</line>
  </example>
</igit>
\end{lstlisting} 
\smallskip
\caption{Yimas IGT in ODIN's XML format}\label{fig:odin_yimas_xml}
\end{figure}


% To clean the source lines, we used the following regular expressions:

% \begin{figure}
% \begin{itemize}
% \item \emph{Cleaner}: Removed (i) all heading and trailing text embedded in square or rounded brackets and (ii) heading double character token ending with bracket or fullstop.
% \begin{itemize}
% \item[(i)]
% \begin{Verbatim}
% ^(?\s?\w{1,5}\s*[):.]\s*
% \end{Verbatim}
% \item[(ii)] 
% \begin{Verbatim}
% [\[\(].{1,}[\]\)]
% \end{Verbatim}
% \end{itemize}
% \item \emph{Cleanest}: Only source lines without punctuation.
% \end{itemize}
% \caption{I don't think we want this as a figure.}
% \end{figure}

% The original version of the ODIN data contains XML files for 1275 languages, while the cleaner version of ODIN contains IGTs for 1042 languages and the cleanest version contains IGTs for 402 languages.  \textcolor{blue}{Maybe we should just remove this next sentence:} The drop from 1275 to 1042 languages was largely because {\color{red} XXXX} XML files from the original ODIN data had used language codes that were not in ISO 639-3 and for {\color{red} XXXX} other files, the \texttt{<igt>...</igt>} tags were missing.

% In future work, we intend to explore more sophisticated methods of cleaning.  For example, we could leverage the fact that the number of hyphens and equal signs in each word should match between the source and gloss lines.  
%However, no method is likely to be foolproof.  
%As people incorporate large sets of IGTs from documentary linguistic projects into a universal corpus, we can expect problems with inconsistencies to grow.

\subsection{Language Codes}
As \newcite{xia2010multilingual} explain, language names do not always suffice to identify languages, since many names are ambiguous. For this reason, sets of language codes exist to more accurately identify languages. We use ISO~639-3\footnote{\url{http://www-01.sil.org/iso639-3/default.asp}} as our standard set of codes, since it aims for universal coverage, and has widespread acceptance in the community. The data from ODIN and the UDHR already used this standard.

To facilitate the standardization of language codes, we have written a python API that can be used to query information about a language or a code, fetching up-to-date information from SIL International (which maintains the ISO~639-3 code set), as well as from Ethnologue.

Wikipedia uses its own set of language codes, most of which are in ISO~639-1 or ISO~639-3.  The older ISO~639-1 codes are easy to recognise, being two letters long instead of three, and can be straightforwardly converted.  However, a small number of Wikipedia codes are not ISO codes at all - we converted these to ISO~639-3, following documentation from the Wikimedia Foundation.\footnote{\url{http://meta.wikimedia.org/wiki/Special_language_codes}}

Omniglot does not give codes at all, but only the language name. To resolve this issue, we automatically converted language names to codes using information from the SIL website.

Some languages have more than one orthography. For example, Mandarin Chinese is written with either traditional or simplified characters; Serbian is written with either the Cyrillic or the Roman alphabet. For cross-linguistic NLP, it could be helpful to have standard codes to identify orthographies, but at present none exist.

\subsection{File Formats}

It is important to make sure that the data we have compiled will be available to future researchers, regardless of how the surrounding infrastructure changes. \newcite{bird2003port} describe a set of best practices for maintaining portability of digital information, outlining seven dimensions along which this can vary. Following this advice, we have ensured that all our data is available as plain-text files, with UTF-8 encoding, labelled with the relevant ISO 639-3 code. Metadata is stored separately. This allows users to easily process the data using the programming language or software of their choice.

To allow access to the data following Abney and Bird's guidelines, as discussed in section \ref{sec:data}, we have written an API, which we distribute along with the data.  Abney and Bird remain agnostic to the specific file format used, but if an alternative format would be preferred, the data would be straightfoward to convert since it can be accessed according to these guidelines.  As examples of functionality, our API allows a user to fetch all sentences in a given language, all sentences from a given source.

\section{Detecting Similar Languages} \label{sec:cluster}

When working on low-resource or endangered languages, computational and documentary linguists face the issue of lack of resources and knowledge about the language. Often, having knowledge about related or similar languages provides useful lexical, syntactic or morphological minimal pairs across languages and also helps in bootstrapping language models in NLP \cite{yarowsky:ngai:2001,xia2007multilingual}.

Language classification can be carried out either: geneologically, mapping languages onto families trees depending on their historical ancenstry \cite{swadesh1952,starostin2010}; or typologically, grouping languages according to typological features \cite{georgi2010wals,daume2009}.

To exemplify the use of the universal corpus for research on low-resource languages, we experimented with automatic detection of similar languages using hierarchical clustering with character ngrams and word unigrams features. Each language is represented by a vector of character bigrams and trigrams and word unigram features from the universal corpus. 
\newline \newline
\noindent \textbf{SHOW FANCY FIGURE OF CLUSTER}
\newline \newline
\textbf{Evaluate the cluster (quantitatively/qualitatively)}


\begin{table*}[h!]
\begin{centering}

    \begin{tabular}{l|ccc|ccc}
    ~        & complete & ~       & ~       & ward    & ~       & ~       \\ \cline{2-7}
    ~        & precision & recall       & f-score       & precision    & recall       & f-score      \\ \hline
    distance & 0.0614   & 0.8565 & \textbf{0.1099} & 0.06140  & 0.8565 & \textbf{0.1099} \\
    maxclust & 0.1925  & \textbf{0.0927} & 0.0692 & \textbf{0.1963} & 0.0905 & 0.0686  \\
    \end{tabular}
\caption{Comparing of clustering algorithms where number of clusters is set to genetic grouping on Ethnologue.}
\end{centering}
\label{tab:cluster}
\end{table*}


\textcolor{blue}{[Use the above figures as a random baseline, and insert the new figures]}


\noindent
see full table in http://goo.gl/bs2vPo





There are many possible metrics to evaluate the quality of a clustering compared to a gold standard. \newcite{amigo2009metrics} propose a set of criteria which a clustering evaluation metric should satisfy, and demonstrate that most popular metrics fail to satisfy at least one of these criteria.  However, they prove that they are satisfied by the BCubed metric, which we adopt for this reason.  To calculate this, we find the induced cluster and gold standard class for each language, and calculate the F-score of the cluster compared to the class.  These F-scores are then averaged across all languages.

As well as performing clustering, we can view this as an information retrieval task: given a language, what are similar languages?  To do this, we found the languages with the closest vectors of n-grams and words.  Since BCubed is calculated averaging across languages and not clusters, we can use exactly the same calculation, using the set of nearby languages in place of a cluster.

In table \ref{tab:cluster}, we give results of our clustering experiments.  The F-scores are much higher than the random baseline, and comparable to the values reported by \newcite{georgi2010wals}, even though we have only used surface features, while they used typological features taken from WALS.  This demonstrates that it possible for cross-linguistic research to be conducted even based on extremely shallow features.

It is also worth noting that precision is higher than recall.  This is perhaps to be expected, given that related languages using wildly differing orthographies will appear to be very different.  Nonetheless, our system is reasonably capable of identifying those languages which are both related and also written similarly.

%\section{Topics related to data release, etc.}


\subsection{Web as Corpus}

\textcolor{blue}{[This might want to go into related work...]}

In recent years, there has been increasing interesting in crawling websites to produce corpora. For example, \newcite{baroni2004bootcat} introduced the BootCaT toolkit to bootstrap specialised corpora, and \newcite{sharoff2006search} built corpora in various languages by issuing queries for random combinations of frequent words to create a balanced corpus not unlike the British National Corpus. \newcite{ferraresi2008ukwac} produced ukWaC, a large-scale English corpus, and \cite{baroni2009wacky} applied a similar method to produce corpora in German, Italian, and French. To build better quality web corpora, \newcite{versley2012quality} introduced the notion of content-sensitive boilerplate detection for cleaning data.

However, despite the amount of text they include, such corpora are usually limited in the number of languages represented. Some projects have emerged which strive for language diversity, such as the Leipzig Corpora Collection (LCC) \cite{biemann2007leipzig}, and COrpora from the Web (COW) \cite{schaefer2012cow}. The LCC currently offers free download of corpora in 117 languages, and dictionaries in many others, bringing the total number of languages up to 230. COW includes corpora in [XX] languages. These collections are certainly commendable, but they currently still fall short of universality - 230 languages represents only 3.3% of the world's languages [check figure].

\newcite{scannell2007crubadan} describes the Crúbadán Project, an attempt to crawl the web for data in a much larger range of languages, including endangered ones. At the time of writing, the number of languages represented has grown to [XXX] However, due to copyright issues, not all of the data is freely available. Motivated by similar copyright concerns, \newcite{brunello2009free} stressed the notion of building free corpora from the web using documents released under Creative Commons licenses.

Unlike the above efforts, we have not performed seed-query web-searching or web-crawling to achieve a balanced resource. Instead, we have focused on a small number of sources which contain data in a wide variety of languages, as described in section \ref{sec:sources}. However, these sources represent a variety of data types which a linguist might encounter, ranging from single-language text with significant non-linguistic markup (Wikipedia) to structured data with detailed linguistic annotations (ODIN).


\subsection{Copyright Issues} \label{sec:copyright}

Corpora copyright remains a grey area and a hobglobin for corpora compilation and distribution (Kilgarriff, 2002)[@Li, please add this reference to the bibliography file]. Often, free access to language data online does not imply the right to download, retain, remix and redistribute. 

ODIN refers itself as a database of IGTs that are returned as a result of querying from the source documents. As such, only the links to the documents and the appropriate citations and not the copies of the source documents are maintained on the ODIN site. However, the ODIN developers have left the users with no specific license/copyrights restriction for readaptation or redistribution.

The Omniglot phrases are free to be used for non-commercial purposes as stated in the FAQ page\footnote{http://www.omniglot.com/faqs.htm} but the rights of the Tower of Babel snippets from different bible sources remains iffy. For example, Omniglot cited www.bibles.org as one of its sources for the Tower of Babel chapter but the bibles.org site stated in fine print that `\emph{Except expressly permitted herein, you may not modify, copy, distribute, decompile, disassemble, reverse engineer, create derivative works, or otherwise use or manipulate the Site or any of its content without our prior written consent}'\footnote{http://bibles.org/pages/legal\#copyright}. [Ga: actually, about copying information, Omniglot says: \emph{'If you would like to copy or re-use any of the language-related articles, please ask the authors for permission. Contact details of the authors can be found at the bottom of the articles in most cases, if they're not there, please contact me, and I'll try to find them for you.'} - I suspect Adger has been given permission for each translation, and I think we would have to ask every single author if we want to re-distribute the data.  For the tower of babel translations, this could end up a nightmare...]

Although users are free to use, copy, modify, merge, publish and distribute Unicode Data Files, [Ga: Do we have a reference for this? Particularly distribution] there is no explicit ownership of the UDHR textfiles from the Unicode website. However, the original source of the UDHR documents from the Office of the High Commissioner for Human Rights clearly states that `\emph{None of the materials provided on this web site may be used, reproduced or transmitted, in whole or in part, in any form or by any means, electronic or mechanical, including photocopying, recording or the use of any information storage and retrieval system, except as provided for in the Terms and Conditions of Use of United Nations Web Sites, without permission in writing from the publisher.}'\footnote{http://www.un.org/en/aboutun/copyright/}. [Ga: Actually, it's not quite so bad - if we look at the terms and conditions, it says: \emph{'The United Nations grants permission to Users to visit the Site and to download and copy the information, documents and materials (collectively, “Materials”) from the Site for the User’s personal, non-commercial use, without any right to resell or redistribute them or to compile or create derivative works therefrom, subject to the terms and conditions outlined below, and also subject to more specific restrictions that may apply to specific Material within this Site.'}\footnote{http://www.un.org/en/aboutun/terms/} Specifically for the UDHR, it says: \emph{'If UDHR translations or materials are reproduced, users should make reference to this website as a source by providing a link.'}\footnote{http://www.ohchr.org/EN/UDHR/Pages/Introduction.aspx}  So, it looks like we're okay to use the UDHR as we currently are, but if we want to redistribute it, we need to get their permission in writing.]

The Wikipedia texts are licensed under Creative Commons Attribution Share-like\footnote{http://creativecommons.org/licenses/by-sa/3.0/legalcode}, which allows redistribution and adaptation. Hence the clean Wikipedia texts we have compiled can be freely redistributed without infringing any copyrights.  We have made this resource publically available.\footnote{To maintain anonymity, we do not give details of how to access the data, but, if accepted for publication, instructions will be included in the print version of this paper.}

%%
%To avoid any copyrights infringement, ... After surveying the rights for so many days, there is no way to redistribute the corpus unless we jumble up the sentences, but if we do that, the "document" level will be lost and that will not be a "good" resource to do NLP for LRL since data is already sparse.
%%



\section{Future Work} \label{sec:future}

Specific suggestions on how to work on resources mentioned in section \ref{sec:related}...


\section{Conclusion} \label{sec:conclusion}

In this paper, we have described the creation of a foundation text for a universal corpus, following the guidelines of Abney and Bird (2010; 2011). To do this, we cleaned and standardised data from several multilingual data sources: ODIN, Omniglot, the UDHR, Wikipedia, and the LCC. The resulting corpus is more easily machine-readable than any of the underlying data sources, and has been stored according to the best practices suggested by \newcite{bird2003port}. [Give a summary statistic of size.] To demonstrate the utility of this resource, we used the data to perform language clustering, which gave promising results. We believe that this is the first corpus of its kind, which we hope will act as a seed corpus for the Human Language Project.

This makes the situation even worse for the computational linguist,
especially one who would like to take a cross-linguistic or
typological approach. We first survey existing efforts to compile


\subsection{General recommendations for future sources?}

\begin{itemize}
\item next steps for DL?
\item next steps for CL?
\item uses of clustering?
\end{itemize}


\subsection{Deliverables/API/something something to say that we have script to access the data easily and CompLing and DocLing people can use them with NLTK}
Other than the clean corpus data, SeedLing comes with a python API to access the data programmatically and it is easily compatible with the analysis tools available from the Natural Language ToolKit (NLTK). In addition, the clustering script used in section X will also be made free available with an open source license.

The API also includes a feature to access information on a language from Ethnologue and/or WALS. This feature queries the Ethnologue and/or WALS site both in real-time when internet connection is available and from a local copy of the information offline. 


\bibliography{references}

\end{document}
