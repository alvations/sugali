
\section{Future Work} \label{sec:future}

Specific suggestions on how to work on resources mentioned in section \ref{sec:related}...


\section{Conclusion} \label{sec:conclusion}

In this paper, we have described the creation of a foundation text for a universal corpus, following the guidelines of Abney and Bird (2010; 2011). To do this, we cleaned and standardised data from several multilingual data sources: ODIN, Omniglot, the UDHR, Wikipedia, and the LCC. The resulting corpus is more easily machine-readable than any of the underlying data sources, and has been stored according to the best practices suggested by \newcite{bird2003port}. [Give a summary statistic of size.] To demonstrate the utility of this resource, we used the data to perform language clustering, which gave promising results. We believe that this is the first corpus of its kind, which we hope will act as a seed corpus for the Human Language Project.

This makes the situation even worse for the computational linguist,
especially one who would like to take a cross-linguistic or
typological approach. We first survey existing efforts to compile


\subsection{General recommendations for future sources?}

\begin{itemize}
\item next steps for DL?
\item next steps for CL?
\item uses of clustering?
\end{itemize}


\subsection{Deliverables/API/something something to say that we have script to access the data easily and CompLing and DocLing people can use them with NLTK}
Other than the clean corpus data, SeedLing comes with a python API to access the data programmatically and it is easily compatible with the analysis tools available from the Natural Language ToolKit (NLTK). In addition, the clustering script used in section X will also be made free available with an open source license.

The API also includes a feature to access information on a language from Ethnologue and/or WALS. This feature queries the Ethnologue and/or WALS site both in real-time when internet connection is available and from a local copy of the information offline. 
