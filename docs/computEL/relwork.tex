\section{Related Work} \label{sec:related}

[There is a lot more stuff to be added here...]
Currently, multilingual corpora efforts have limited coverage in the number of languages and the number of language families represented. For instance, the OPUS corpus \cite{tiedemann2012opus} covers over 90 languages, only 1.27\% of the world's languages; The Leipzig Corpora Collection\footnote{corpora.uni-leipzig.de} \cite{biemann2007leipzig} contains corpora and dictionaries in 230 languages, 3.24\% of the world's languages. Even corpora that boast of linguistic diversity include only a small number of language families, e.g. the linguistically diverse NTU-Multilingual Corpus \cite{tan2011ntu} covers only 7 out of 136 language families. Producing a universal corpus requires the linguistic community to merge existing corpora and provide an ubiquitous access interface. To compile the foundation text for the Universal Corpus, we crawled and cleaned web data that contains multilingual texts and merged them with existing corpora collections to form the foundation text for the Universal Corpus.

