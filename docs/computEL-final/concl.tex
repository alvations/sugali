\section{Conclusion and Outlook} \label{sec:conclusion}

In this paper, we have described the creation of SeedLing, a foundation text for a universal corpus, following the guidelines of Abney and Bird (2010; 2011). To do this, we cleaned and standardised data from several multilingual data sources: ODIN, Omniglot, the UDHR, Wikipedia. The resulting corpus is more easily machine-readable than any of the underlying data sources, and has been stored according to the best practices suggested by \newcite{bird2003port}. At present, SeedLing has data from 19\% of the world's languages, covering 72\% of language families. We believe that a corpus with such high degrees of language diversity, uniformity and cleanliness of data format, and ease of access provides an excellent seed for the universal corpus. It is our hope that taking steps toward creating this resource will spur both further data contributions and interesting computational research with cross-linguistic or typological perspectives; we have here demonstrated SeedLing's utility for such research by using the data to perform language clustering, with promising results.

SeedLing (and the API) is currently available via a GitHub repository.\footnote{\url{https://github.com/alvations/SeedLing}} We have yet to fully address questions of long-term access, and we welcome ideas or collaborations along these lines.
