\section{Introduction} \label{sec:intro}

At the time of writing, 7105 living languages are documented in Ethnologue,\footnote{\url{http://www.ethnologue.com}} but \newcite{simons2011update} calculated that 37\% of extant languages were at various stages of losing transmisson to new generations. Only a fraction of the world's languages are well documented, fewer have machine-readable resources, and fewer again have resources with linguistic annotations \cite{maxwell2006annotation} - so the time to work on compiling these resources is now.

Several years ago, Abney and Bird \shortcite{abney2010universal,abney2011data} posed the challenge of building a Universal Corpus, naming it the Human Language Project. Such a corpus would include data from all the world's languages, in a consistent structure, facilitating large-scale cross-linguistic processing. The challenge was issued to the computational linguistics community, from the perspective that the language processing, machine learning, and data manipulation and management tools well-known in computational linguistics must be brought to bear on the problems of documentary linguistics, if we are to make any serious progress toward building such a resource. The Universal Corpus as envisioned would facilitate broadly cross-lingual natural language processing (NLP), in particular driving innovation in research addressing NLP for low-resource languages, which in turn supports the language documentation process. 
%\textcolor{blue}{[cite our AL work? is this too strong? also, run-on sentence. also, perhaps mention this workshop, which has precisely these goals]}

We have accepted this challenge and have begun converting existing resources into a format consistent with Abney and Bird's specifications. We aim for a collection of resources that includes data: (a) from as many languages as possible, and (b) in a format both in accordance with best practice archiving recommendations and also readily accessible for computational methods. Of course, there are many relevant efforts toward producing cross-linguistic resources, which we survey in section \ref{sec:related}. To the best of our knowledge, though, no existing effort meets these two desiderata to the extent of our corpus, which we name SeedLing: a seed corpus for the Human Language Project.

To produce SeedLing, we have drawn on four web sources, described in section \ref{sec:sources}. To bring the four resources into a single common format and data structure (section \ref{sec:structure}), each required different degrees and types of cleaning and standardisation. We describe the steps required in section \ref{sec:case_studies}, presenting each resource as a separate mini-case study. We hope that the lessons we learned in assembling our seed corpus can guide future resource conversion efforts. To that end, many of the resources described in section~\ref{sec:related} are candidates for inclusion in the next stage of building a Universal Corpus.

We believe the resulting corpus, which at present covers 1451 languages from 105 language families, is the first of its kind: large enough and consistent enough to allow broadly multilingual language processing. To test this claim, we use SeedLing in a sample application (section \ref{sec:cluster}): the task of language clustering. With no additional pre-processing, we extract surface-level features (frequencies of character n-grams and words) to estimate the similarity of two languages. Unlike most previous approaches to the task, we make no use of resources curated for linguistic typology (e.g. values of typological features as in WALS \cite{wals}, Swadesh word lists). Despite our approach being highly dependent on orthography, our clustering performance matches the results obtained by \newcite{georgi2010wals} using typolological features, which demonstrates SeedLing's utility in cross-linguistic research.